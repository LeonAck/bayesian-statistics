\documentclass[report]{subfiles}

\begin{document}


\section*{Guide lines}
\label{sec:time-line}

\subsection*{General info}
\label{sec:general}


\begin{enumerate}
\item  Students in groups of three write a report. We rely on you forming groups.
\item Each group has to choose a method (from the primary book, or from one of the other sourcesbelow) which the members of the group are interested in learning. A case based on a company at which one of the student is doing a master thesis would be an excellent choice.
Otherwise, the group has to find a problem on the internet with data to which to apply their method of choice.
\item Each group has to review an intermediate and near to final version of the report of an other group.
\item The core of the report is at most 10 papes. Appendices are not included, but when long, we (=avv and nvf) will skip them.
\item Each group chooses another group to exchange feedback.
  Each group gives and receives feedback twice, on an intermediate and near final version of the report.
\end{enumerate}

\subsection{Methods}
\label{sec:methods}


It is sufficient to understand the basic methods, and, as a general principle, it is best to start with simple methods based on what you have learned during the course.
Once you understand the simple methods, you can move on to more difficult ones.
If you use such harder methods, do not forget to explain in your report the simple methods you studied first, and what motivated you to dive into more challenging methods.

We do not give grades based on the performance of your methods; the grading is more done based on your explanation of the methods and \emph{why} you think they perform good or bad. Or, after you did the analysis, why your methods does, or doesn't work, and what tests you used to convince yourself about your claims.

If (part of) your report is based on an project for a company, don't feel obliged that your method(s) `must work'.
The best advice is an honest advice.
If you cannot find good methods to predict what they would like to see, just say so.
(Let's reverse it, suppose your predictor works lousy, but you say it gives fantastic results.
This is a recipe for disaster.)
You already `deliver' by saying that some things cannot be done.

You don't have to explain each method you use in your report.
However, you should explain just one such method in considerable detail.
The motivation is this.
If you can explain one method well, we trust that you can also explain other methods well.
So why do we mind about details?
Well, after finishing your master, you have to be able to explain such methods at various level of detail to a board of managers.
When there are managers with a math, computer science, engineering background, they might want to understand in very much detail what you built.
For instance, my (=nvf) manager at Quintiq was Victor Allis. (You can find him on wikipedia.)
Whenever somebody proposed some algorithm, he always wanted to really understand things in detail.
And if you could not convince him during a meeting, you could go back to your desk, and start thinking again.
There are also other types of managers that ask nasty questions,  just to check whether you did your work thoroughly.
People do not invest several millions of euros, just because you say things are ok.
You have to convince them, and show that you know what you are talking about.
And if you fail at that skill a few times in row, you will have a problem, if you want to keep your job at least.

Your report should show that you know what you are talking about.
We realize that this is a hard challenge.
But both in business and academia, people don't like to read long reports.
So, train and learn. Again, there is no way around for you (unless you decide to become a shepherd, for instance).

\subsection*{Time lines}
Here is a time line to help you organize your work.

\begin{enumerate}
\item Friday 23/4, 17h: topic choice uploaded to nestor.
  Keep it short, and use this as a start for your intro.
  Please include some motivation about why you chose this topic, why you think it is suitable, and feasible within the course.
  The idea is that we (= avv and nvf) can give you some feedback about whether the topic is good, too simple or too hard.
  Include a phone number on the first page of your report of at least one of you so that we can contact you to give verbal feedback.
\item  Friday 23/4: Aim to have \cref{sec:data-description} written. No upload to nestor.
\item   Friday 30/4, 17h: research proposal uploaded to nestor. \cref{sec:methods-validation} written.  Send it to the other group for feedback.
\item Tuesday 4/5, 17h: intermediate feedback uploaded to nestor. Feedback on intermediate report sent back to the other group.
\item  Friday 7/5 17h:   Mutual feedback discussed.
\item   Friday 28/5 17h: Final version 0.1 uploaded to nestor. Methods implemented and tested, and report nearly finished.
  Ensure you dealt with the feedback.
  Send report to feedback group.
\item Tuesday 1/6, 17h: Final  feedback report uploaded to nestor. Plan a feedback session.
\item  Friday 4/6, 23h59: Final report uploaded to on nestor.
  Ensure again that you dealt with the final feedback.
\end{enumerate}

\begin{enumerate}
\item Include the tex files in your uploads to nestor.
\item Also include the feedback, in the appendix.  We don't include your feedback in the page count of your own report.
\end{enumerate}


\end{document}
